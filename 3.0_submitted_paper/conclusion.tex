\section{Conclusion and Future Work}

In this paper, we carry out an empirical study to explore the evidences in favor of or against multi-lingual Q\&A sites.
Our study examines not only the user perspective, including user registration, posting behavior and reputation score, but also the knowledge perspective, including tag uniqueness and tag correlation between multi-lingual sites, and explicit cross-site links created by users and implicit cross-site links that users are not aware of.
Our study results help to resolve the arguments about the risk of community split and the knowledge needs and interests in non-English language.
We also confirm the validity of the concern about knowledge fragmentation and duplication, and provide a promising cross-site retrieval method for mitigating this issue.

Our evidence-based study method provides a guideline to study the pros and cons of multi-lingual Q\&A sites, and our findings shed the light on the development of multi-lingual Q\&A sites.
In the future, based on our findings, we will develop tools to help bridge the gap across multi-lingual Q\&A sites, such as recommending duplicate or related posts on English Stack Overflow to Russian Stack Overflow questions, and developing domain-specific deep learning based machine translation model for Q\&A discussions.

\begin{comment}
show the influence of the launch of multi-lingual Stack Overflow.
We first spot the fierce debate of the decision of building multi-lingual Stack Overflow, and obtain the pros and cons by analysing users' comments under the announcement of the new sites.
To validate the benefits and concerns, we carry out quantitative empirical study between English Stack Overflow and Russian Stack Overflow from both user and content aspects.
The results show the value of existence of multi-lingual Stack Overflow and resolve some concerns like community split of the main site.
Some concerns are also confirmed such as knowledge fragmentation and content duplication.
These discoveries can play an important criteria to guide the future development of not only multi-lingual Stack Overflow, but also other Q\&A sites like Quora.
\end{comment}



\par

%Last but not least, we will implement a new recommending tool that will be deployed on Stack Overflow or some other Q\&A websites to assist non-English language speakers to utilize the knowledge in the English Q\&A site better.